\section{The Pisticci Farm Project}\label{the-pisticci-farm-project}

The Pisticci Farm Project is a project of the Upper Manhattan restaurant
\href{http://www.pisticcinyc.com/}{Pisticci}. The project has three
elements:

\begin{itemize}
\itemsep1pt\parskip0pt\parsep0pt
\item
  \emph{engaging} the \textbf{restaurant} staff and patrons
\item
  creating \emph{zero waste} through
  \hyperref[compost]{\textbf{composting}} within the city
\item
  \emph{producing} high-quality vegetables for Pisticci on the
  \hyperref[farm]{\textbf{multi-site farm}}
\item
  fully utilising available space through
  \hyperref[mushrooms]{\textbf{mushroom production}}
\end{itemize}

This document outlines the technologies and processes involved in the
composting and farming elements of the project.

\hyperdef{}{compost}{\subsection{Compost}\label{compost}}

The discarded organic material from Pisticci restaurant are separated at
source (ie. the kitchen, bar, etc.) from other `waste'. These materials
are composted using an oxygenated hot composting method to ensure rapid
decomposition, no unpleasant odours and the creation of a high quality
fertiliser and biological innoculum for the soil at the
\hyperref[farm]{farm}.

\subsubsection{Aerated Static Pile (ASP)
Composting}\label{aerated-static-pile-asp-composting}

Pisticci is using the aerated static pile (ASP) method of composting as
developed by Peter Moon of \href{http://www.o2compost.com}{O2 Compost}.

The Pisticci ASP system using includes three enclosed boxes which are
bottem-fed pressurised air through \emph{100 mm} (\emph{4 in}) from a
blower mounted above the bins.

\paragraph{Organic Material Collection
Process}\label{organic-material-collection-process}

\begin{itemize}
\itemsep1pt\parskip0pt\parsep0pt
\item
  Organic materials are collected in white plastic bags by the
  restaurant staff and deposited in wheelie-bins next to the compost
  bins.
\end{itemize}

\paragraph{Composting Processes}\label{composting-processes}

On a week-by-week basis, the Pisticci ASP composting process
incorporates the following steps:

\textbf{Establishing a new Pile}

\begin{enumerate}
\def\labelenumi{\arabic{enumi}.}
\itemsep1pt\parskip0pt\parsep0pt
\item
  Ensure that all pipes in the manifold in the bottom of the bin are
  present and properly fitted together.
\item
  Add woodchips to just cover the pipes and thoroughly moisten the
  woodchips.
\item
  Open the valve above the bin to ensure that air will flow into the
  pile while composting proceeds.
\end{enumerate}

Organic materials may now be added.

\textbf{Ongoing Addition of Organic Materials}

\begin{enumerate}
\def\labelenumi{\arabic{enumi}.}
\itemsep1pt\parskip0pt\parsep0pt
\item
  Use the composting fork to mix through any previously added materials
  and add water from the hose to bring the moisture levels up to
  approximately \hyperref[50moisture]{50\% moisture}.
\item
  Empty two or three white bags of organic materials into the bin
  currently in use. Attempt to mix bags which are mostly vegetable waste
  (\hyperref[highnitrogen]{high nitrogen materials}) and bags which are
  mostly napkins / dry `\hyperref[highcarbon]{high carbon}' materials.
\item
  Mix these materials together in the bin using the composting fork.
\item
  Use the bucket to carry several buckets of woodchips from the woodchip
  storge bin to the compost bin to cover the material from the bags.
\item
  Mix the woodchips into the other materials using the composting fork.
\item
  Allow the materials to sit in the bin until the next addition to
  absorb moisture before adding any additional water (see point 1
  above).
\end{enumerate}

\textbf{When a Bin is Full}

\begin{enumerate}
\def\labelenumi{\arabic{enumi}.}
\itemsep1pt\parskip0pt\parsep0pt
\item
  When a bin is full, make sure the moisture levels are appropriate (ie.
  \hyperref[50moisture]{50\% moisture}), add a layer of woodchip over
  the top layer and cover with geotextile fabric / weed matting.
\item
  ensure the valve for the air is in the fully open position.
\item
  Leave to mature for as long as possible before the compost is taken up
  to the farm
\item
  Compost will be taken to the farm when two bins are full and one bin
  is one quarter to half-way full.
\end{enumerate}

\textbf{Transporting Compost to the Farm}

When two bins are full and one bin is one quarter to half-way full, the
most mature pile should be transported to the farm.

\begin{enumerate}
\def\labelenumi{\arabic{enumi}.}
\itemsep1pt\parskip0pt\parsep0pt
\item
  Position the truck somewhere near to the front of the restaurant which
  will allow for wheelbarrow access via the gate at the side of the
  building.
\item
  Lift the front panel of the bin to be removed so the the bolts which
  secure it are lifted from their holes and set the front panel aside.
\item
  Using a wheelbarrow, transport the compost to the truck until the bin
  is empty.
\item
  Clean all areas and close the bin by returning the front panel to its
  place.
\end{enumerate}

\paragraph{System Specifications}\label{system-specifications}

\begin{itemize}
\itemsep1pt\parskip0pt\parsep0pt
\item
  \textbf{The Bins} are 5' long, 4' high and 3' wide to fit within the
  small alleyway behind the restaurant.
\item
  \textbf{The Blower}
\item
  \textbf{The Timer} is set to be on for 2 minutes every 30 minutes to
  ensure the piles remain oxygenated.
\end{itemize}

\hyperdef{}{farm}{\subsection{Farm}\label{farm}}

The Pisticci Farm Project is a multi-site micro farm. At the time of
writing the farm consists of two sites:

\begin{itemize}
\itemsep1pt\parskip0pt\parsep0pt
\item
  43 Old Post Rd South, Croton-on-Hudson, NY (OPRS); and
\item
  716 Kitchawan Rd, Ossining, NY (KITCH).
\end{itemize}

The \textbf{Old Post Rd South} site was developed and farmed in 2015.
The \textbf{Kitchawan} site was secured at the end of 2015 and will be
developed and farmed in addition to the OPRS site in 2016.

\subsubsection{Old Post Rd South}\label{old-post-rd-south}

The Old Post Rd South site is the flagship site of the Pisticci Farm
Project. It consists of indoor and outdoor growing space. There are two
large old glass greenhouses which have been restored and developed for
all sesason production of vegetables. The 2015 growing season was used
for diverse crop production with a focus on fast growing, high value
crops, variety trials and winter greens production. In 2016, this site
will be used primarily for all season greens production and greenhouse
production of tomatoes, peppers, eggplant and cucumbers.

\paragraph{The Outdoor Growing Area}\label{the-outdoor-growing-area}

The outdoor growing area consists of 27 beds of varying sizes in three
blocks (Lower, Middle, Top). The beds are all 30 inches wide with 18
inch paths between them and vary in size from 10 to 40 feet with a total
of 620 linear feet of bed space (1550 ft2).

\paragraph{The Greenhouses}\label{the-greenhouses}

In 2015, the indoor space consisted of 12 beds across two greenhouses.
The beds ranged in size from 29 to 60 feet with a total of 517 linear
feet of bed space (1292 ft2). In early 2016, these indoor beds were
delineated with 6" x 2" boards to create a total of 14 beds 29 inches
wide with 11 inch intermediate paths. This new configuration has a total
of 550 bed feet (1630 ft2).

\textbf{Hydronic Heating} is used in the greenhouses to allow for all
season production. The original soil is separated from the imported soil
and compost by 2 inches of foam insulation. PEX tubing runs over the top
of the insulation at 12 inch spacing. The PEX tubing is covered with
between 12 and 18 inches of topsoil mixed with compost. This soil is
kept at a minimum of 50°F (10°C) to allow for ongoing plant growth
during the cooler month.

\subsubsection{Kitchawan}\label{kitchawan}

Kitchawan Farm is a diversified working farm with a focus on rough
housing of horses. The Pisticci Farm project leases a small area for
vegetable production. In Spring of 2016, this area will be developed
into sets of 30 inch wide beds. This site will be used for crops n

\subsubsection{Crops}\label{crops}

\paragraph{Priority Crops for 2016}\label{priority-crops-for-2016}

\begin{longtable}[c]{@{}llllr@{}}
\toprule\addlinespace
Type & Crop & Varieties & DTM & Location
\\\addlinespace
\midrule\endhead
green & \textbf{\emph{arugula}} & astro & 21-30 & OPR
\\\addlinespace
green & \textbf{\emph{kale}} & toscano & 35-40 & KITCH
\\\addlinespace
green & \textbf{\emph{lettuce}} & salanova & 55 & KITCH
\\\addlinespace
green & \textbf{\emph{mesclun mix}} & 5 star; all star & 28-30 & OPR
\\\addlinespace
green & \textbf{\emph{dandelion}} & clio, garnet stem & 35-48 & OPR
\\\addlinespace
green & \textbf{\emph{escarole}} & eros & 45 & OPR
\\\addlinespace
green & \textbf{\emph{beet greens}} & bulls blood; early wonder; red
devil & 35 & KITCH
\\\addlinespace
cucurbit & \textbf{\emph{summer squash}} & zephyr; goldmine; safari;
slik pik & 50-54 & KITCH
\\\addlinespace
cucurbit & \textbf{\emph{pattypan squash}} & sunburst; g-start & 50-52 &
KITCH
\\\addlinespace
cucurbit & \textbf{\emph{cucumber}} & katrina & 48 & OPR
\\\addlinespace
solanum & \textbf{\emph{pepper}} & carmen; sprinter; sympathy; moonset &
60 & GH; KITCH
\\\addlinespace
solanum & \textbf{\emph{tomato}} & taxi; rebelski; manero; beorange &
70-75 & GH
\\\addlinespace
solanum & \textbf{\emph{tomato (cherry)}} & santorange; favorita; yellow
pear & 58-70 & GH
\\\addlinespace
solanum & \textbf{\emph{eggplant}} & nadia; orient charm; jaylo & 62-67
& GH; KITCH
\\\addlinespace
herb & \textbf{\emph{basil}} & nufar; neopolitano & 77 & GH; KITCH
\\\addlinespace
root & \textbf{\emph{beet}} & baby beet; boulder; red ace & 40-50 &
KITCH
\\\addlinespace
root & \textbf{\emph{radish}} & d'avignon & 21 & KITCH
\\\addlinespace
\bottomrule
\end{longtable}

\subsubsection{Crop Descriptions}\label{crop-descriptions}

The following includes cultivation notes on each of the above crops if
they were successfully grown in 2015. If crops were not grown in 2015 or
the cultivation description is well detailed elsewhere then references
are given.

\paragraph{arugula}\label{arugula}

\begin{quote}
\textbf{Propagation:} Direct Sown

\textbf{Seeder:} Glasser (small bore)

\textbf{Rows:} 7-9 (more rows makes plants `stemmy')

\textbf{Harvest:} Use \texttt{Serrated Greens Knife (blue handle)}

\textbf{Store:} Coolroom
\end{quote}

\paragraph{basil}\label{basil}

\begin{quote}
\textbf{Propagation:} 72 cell tray

\textbf{Rows:} 3

\textbf{Spacing:} 12"

\textbf{Harvest:} Cut stems down to node with \texttt{secateurs}

\textbf{Store:} Coolroom
\end{quote}

\paragraph{beets}\label{beets}

Beets were not grown as a root crop in 2015. Refer to
\texttt{The Market Gardener} or other reference for cultural details.

\paragraph{beet greens}\label{beet-greens}

\begin{quote}
\textbf{Propagation:} Direct Sown

\textbf{Seeder:} Earthway (chard plates)

\textbf{Rows:} 7

\textbf{Harvest:} Use \texttt{Serrated Greens Knife (blue handle)}

\textbf{Store:} Coolroom
\end{quote}

\paragraph{cucumber}\label{cucumber}

Cucumbers were not grown in 2015. Refer to \texttt{The Market Gardener}
or other reference for cultural details.

\paragraph{dandelion}\label{dandelion}

\begin{quote}
\textbf{Propagation:} 72 cell tray

\textbf{Rows:} 5

\textbf{Spacing:} 6"

\textbf{Harvest:} Use \texttt{Serrated Greens Knife (blue handle)}

\textbf{Store:} Coolroom
\end{quote}

\paragraph{eggplant}\label{eggplant}

\begin{quote}
\textbf{Propagation:} 72 cell tray transplanted to 4" pot

\textbf{Rows:} 1

\textbf{Spacing:} 18"

\textbf{Harvest:} Cut stems above fruit with \texttt{secateurs}

\textbf{Store:} Basement
\end{quote}

\paragraph{escarole}\label{escarole}

\begin{quote}
\textbf{Propagation:} 72 cell tray

\textbf{Rows:} 4

\textbf{Spacing:} 8"

\textbf{Harvest:} Use \texttt{Serrated Greens Knife (blue handle)}

\textbf{Store:} Coolroom
\end{quote}

\paragraph{kale}\label{kale}

\emph{baby leaf}

\begin{quote}
\textbf{Propagation:} Direct Sown

\textbf{Seeder:} Glasser (small bore)

\textbf{Rows:} 7

\textbf{Harvest:} Use \texttt{Serrated Greens Knife (blue handle)}

\textbf{Store:} Coolroom
\end{quote}

\emph{large leaf}

\begin{quote}
\textbf{Propagation:} 72 cell tray

\textbf{Rows:} 4

\textbf{Spacing:} 8"

\textbf{Harvest:} twist off mature leaves

\textbf{Store:} Coolroom
\end{quote}

\paragraph{lettuce (salanova)}\label{lettuce-salanova}

\begin{quote}
\textbf{Propagation:} 72 cell tray

\textbf{Rows:} 4

\textbf{Spacing:} 8"

\textbf{Harvest:} Cut off outside leaves with
\texttt{lettuce field knife (yellow handle)}

\textbf{Store:} Coolroom
\end{quote}

\paragraph{mesclun mix}\label{mesclun-mix}

\begin{quote}
\textbf{Propagation:} Direct Sown

\textbf{Seeder:} Glasser (large bore)

\textbf{Rows:} 9

\textbf{Harvest:} Use \texttt{Serrated Greens Knife (blue handle)}

\textbf{Store:} Coolroom
\end{quote}

\paragraph{pattypan squash}\label{pattypan-squash}

\begin{quote}
\textbf{Propagation:} 72 cell tray transplanted to 4" pot

\textbf{Rows:} 1

\textbf{Spacing:} 18"

\textbf{Harvest:} twist off or cut off fruits at base of stem

\textbf{Store:} Basement
\end{quote}

\paragraph{pepper}\label{pepper}

\begin{quote}
\textbf{Propagation:} 72 cell tray transplanted to 4" pot

\textbf{Rows:} 1

\textbf{Spacing:} 18"

\textbf{Harvest:} twist off or cut off fruits at base of stem

\textbf{Store:} Basement
\end{quote}

\paragraph{radish}\label{radish}

Radishes were not grown in 2015. Refer to \texttt{The Market Gardener}
or other reference for cultural details.

\paragraph{summer squash}\label{summer-squash}

\begin{quote}
\textbf{Propagation:} 72 cell tray transplanted to 4" pot

\textbf{Rows:} 1

\textbf{Spacing:} 18"

\textbf{Harvest:} twist off or cut off fruits at base of stem

\textbf{Store:} Basement
\end{quote}

\paragraph{tomato}\label{tomato}

\begin{quote}
\textbf{Propagation:} 72 cell tray transplanted to 4" pot

\textbf{Rows:} 1

\textbf{Spacing:} 18"

\textbf{Harvest:} twist off or cut off fruits at base of stem

\textbf{Store:} Basement
\end{quote}

Refer to \texttt{The Market Gardener} or other reference for details of
trellising system and offset training.

\subsubsection{Nursery}\label{nursery}

Pisticci Full Circle Farm produces all of its own seedlings. All seed to
date has been purchased from Johnny's Selected Seed. Potting medium is
made at the farm.

\paragraph{Potting Medium}\label{potting-medium}

Potting soil is usually made up in batches of one wheelbarrow with the
major ingredients being measured using a 5 gallon bucket. The
\textbf{potting soil} consists of: * 2 buckets sifted compost * 2
buckets sifted coir * 1 bucket sifted bed soil * 3 tbsp fish hydrolysate
* 1 tsp solu-kelp * 3 tbsp EM * 2 buckets water

\emph{Method}

\begin{enumerate}
\def\labelenumi{\arabic{enumi}.}
\itemsep1pt\parskip0pt\parsep0pt
\item
  mix fish hydrolysate, solu-kelp and EM in water
\item
  add coir to water and let stand until expanded; add extra water if
  required
\item
  sift all other ingredients and mix together thoroughly
\item
  store in galvanised bin, preferably for a week or more before use.
\end{enumerate}

\subsubsection{Fertility}\label{fertility}

Soil is at the heart of any vegetable growing system with integrity. The
Pisticci Farm project utilises a number of strategies for developing and
maintaining a healthy soil for the production of healthy nutritios
plants. These include:

\begin{itemize}
\itemsep1pt\parskip0pt\parsep0pt
\item
  composting
\item
  soil testing and ammending for mineral balance
\item
  biological innocula / biofertiliser
\item
  minimal tillage
\item
  fertigation
\end{itemize}

\paragraph{Soil Testing and Ammending for Mineral
Balance}\label{soil-testing-and-ammending-for-mineral-balance}

Soil testing has been carried out to determine how to ammend soil to
achieve a balance of minerals for ideal plant health. The principal tool
used to determine the `ideal' is the set of worksheets created by Steve
Solomon and Erica Reinheimer (available at
\url{http://www.newsociety.com/var/storage/blurbs/IntelligentGardener-Worksheets.pdf}).
This process is well described in ``The Intelligent Gardener'' by Steve
Solomon.

\subparagraph{Recommended Soil Treatments for
2016}\label{recommended-soil-treatments-for-2016}

TODO TABLE

\paragraph{Biological Innocula}\label{biological-innocula}

The principal form of biological activation / innoculation of soil and
leaf surfaces is through the use of EM; effective microorganisms.

\textbf{Activated EM} is created through the following process:

\emph{Ingredients and Equipment}

\begin{enumerate}
\def\labelenumi{\arabic{enumi}.}
\itemsep1pt\parskip0pt\parsep0pt
\item
  EM•1® --- 1 gallon
\item
  Unsulfured Blackstrap Molasses --- 1 gallon
\item
  water (preferably warm / \textasciitilde{} 110°F (43°C).
\item
  pH test papers with range to 3.5
\item
  airtight container and airlock
\end{enumerate}

\emph{Method}

\begin{enumerate}
\def\labelenumi{\arabic{enumi}.}
\itemsep1pt\parskip0pt\parsep0pt
\item
  Mix ingredients in the container.
\item
  Check the initial pH with pH paper.
\item
  Put on lid and airlock
\item
  Ferment at room temperature for 3-5 days.
\item
  Some time between days 3 and 5, remove the lid and check the pH of the
  liquid using pH paper. If the pH is 3.8 or below, allow the
  fermentation to complete for an additional 5-7 days. Toward the end of
  the fermentation, check the smell of the product. It should have some
  alcohol smell, some white flakes on it and look and smell similar to
  the original EM•1®. If all these are true, it is ready to use.
\end{enumerate}

\textbf{Activated EM} is used in a range of applications at Pisticci
Full Circle Farm:

\begin{itemize}
\itemsep1pt\parskip0pt\parsep0pt
\item
  \textbf{Foliar Spray} --- 3 tbsp (1.5 oz) of \emph{activated EM} to
  1.5 gallons of water --- apply to leaf surfaces with sprayer

  \begin{enumerate}
  \def\labelenumi{\arabic{enumi}.}
  \itemsep1pt\parskip0pt\parsep0pt
  \item
    other soluble fertilisers (solukelp, fish hydrolysate, blood meal)
    may be added
  \end{enumerate}
\item
  \textbf{Potting Media} -- add 3 tbsp to the water used to hydrate the
  coir block or add to existing potting mix
\item
  \textbf{Soil Drench} --- 3 tbsp (1.5 oz) of \emph{activated EM} to 1.5
  gallons of water --- apply to soil with sprayer or through venturi.
\end{itemize}

\paragraph{Fertigation}\label{fertigation}

A Mazzei venturi injection system has been installed in the water line
that preceeds the solenoid valves which control each of the irrigation
stations. This allows soluble nutrient and biological innocula to be
applied through the irrigation water given to crops.

The ideal use of this system is still being worked out. The current
regime is as follows:

\emph{Application Rate}

\begin{itemize}
\itemsep1pt\parskip0pt\parsep0pt
\item
  EM --- ( 40 gallons / acre / year) \textbf{0.45 oz / 100 ft2 / 14
  days}
\item
  Solukelp --- (312 oz / acre / year) \textbf{0.03 oz / 100 ft2 / 14
  days}
\item
  Neptune's Harvest Fish Fertilizer --- (12 gallons / acre / year)
  \textbf{0.14 oz / 100 ft2 / 14 days}
\end{itemize}

\emph{Western Greenhouse} (400 sq ft bed area)

\begin{itemize}
\itemsep1pt\parskip0pt\parsep0pt
\item
  EM --- 6 tbsp / 14 days
\item
  Solu-Kelp --- 1 tsp / 14 days
\item
  Neptune's Harvest Fish Fertilizer --- 2 tbsp / 14 days
\end{itemize}

\emph{Eastern Greenhouse} (900 sq ft bed area)

\begin{itemize}
\itemsep1pt\parskip0pt\parsep0pt
\item
  EM --- 9 tbsp / 14 days
\item
  Solu-Kelp --- 1.5 tsp / 14 days
\item
  Neptune's Harvest Fish Fertilizer --- 2.5 tbsp / 14 days
\end{itemize}

 \#\# Mushroom Production

Not all the space available on the Pisticci Farm Project sites is
suitable for plant production. In particular, there remain underutilised
indoor spaces. To begin to make the best use of these spaces mushroom
cultivation in a controlled environment is being developed.

\subsubsection{The Technology}\label{the-technology}

The mushroom grow room is in the basement of the Old Post Rd South site
and consists of a basic timber framed room insulated with 2" foam board.
The roof is insulated with fibreglass insulation sealed from moisture
with clear plastic.

The controls and monitoring for the room have been set up largely
following the \texttt{mycodo} project created by Kyle Gabriel
(\url{http://kylegabriel.com/projects/2015/04/mushroom-cultivation-revisited.html}).

A
\href{https://www.raspberrypi.org/products/raspberry-pi-2-model-b/}{raspberry
pi} forms the core of the control system. The \texttt{pi} reads the
temperature and humidity sensor and, using a
\href{http://www.csimn.com/CSI_pages/PIDforDummies.html}{PID
controller}, switches fans, a heater and humidifier depending on the
environmental conditions.

\paragraph{Simple Oyster Mushroom
Production}\label{simple-oyster-mushroom-production}

We have aimed for the simplest (and dirtiest) production method that
will still produce a product that is valuable to the Pisticci Kitchen.
In the proof-of-concept stage we used a cold `pasteurisation' method.
The method currently under development will use heat pasteurisation
where water heating is provided by an hydronic coil heated by an
efficient gas water heater.

\emph{Cold Pasteurisation}

\begin{enumerate}
\def\labelenumi{\arabic{enumi}.}
\itemsep1pt\parskip0pt\parsep0pt
\item
  Part fill a 20 gallon drum with water
\item
  add 10 oz (300 g) of hi calcium hydrated lime and mix well
\item
  add water to approximately 15 gallons
\item
  fill mesh sack(s) with straw and slowly push the sacks into the barrel
  until the barrel is holding as much straw as possible. The lime
  solution should cover the sack(s).
\item
  leave for 12 hours
\item
  remove sacks from barrel and allow to drain
\item
  remove straw from sacks and spread out to dry
\end{enumerate}

\emph{Hot Pastuerisation}

\begin{enumerate}
\def\labelenumi{\arabic{enumi}.}
\itemsep1pt\parskip0pt\parsep0pt
\item
  add water to barrel to fill to approximately 3/4 of the barrel's
  volume
\item
  turn on water heater and allow water to reach 160-170°F (70-75°C).
\item
  fill mesh sack(s) with straw and slowly push the sacks into the barrel
  until the barrel is holding as much straw as possible.
\item
  maintain the water temperature in the 160 - 170°F range for 1 hour
\item
  remove sacks from barrel and allow to drain
\item
  remove straw from sacks and spread out to dry
\end{enumerate}

Our mushroom production system uses hepa-filtered polyethelene bags for
the colonisation and fruiting stage.

\emph{Innoculation}

\begin{enumerate}
\def\labelenumi{\arabic{enumi}.}
\itemsep1pt\parskip0pt\parsep0pt
\item
  Take a bag of fully colonised innoculated gri and break u te `cake'
  into individual grains.
\item
  Add two hadfulls of straw to a hepa-filtered polyethelene bag.
\item
  Sprinkle a `pinch' of innoculated grain onto the straw and repeat this
  process until the bag is full enough to still allow for heat sealing
\item
  heat seal the bag
\item
  repeat until all the pasteurised straw is used and store bags in a
  dark space for colonisation
\end{enumerate}

\emph{Fruiting}

After 5-6 weeks the bags should be sufficiently colonised to be moved to
the fruiting room.

\begin{enumerate}
\def\labelenumi{\arabic{enumi}.}
\itemsep1pt\parskip0pt\parsep0pt
\item
  When the straw in the bags is fully colonised (covered in white
  mycelium) move the bags to the fruiting room
\item
  Cut cross-shaped slits into the bags using a razor knife.
\item
  Spray the bags with water if they appear very dry.
\item
  Leave in fruiting room until harvestable mushrooms develop.
\item
  Harvest fruiting bodies by twisting off the mature cluster (make sure
  to do this before the mushrooms release their spores to reduce the
  chance of creating unsafe conditions in the fruiting room).
\end{enumerate}

\paragraph{Controlled Environments Using Open Source Software and
Inexpensive Microcontrollers and DIY
Electronics}\label{controlled-environments-using-open-source-software-and-inexpensive-microcontrollers-and-diy-electronics}

\subsubsection{The Process}\label{the-process}

\subsection{Suppliers}\label{suppliers}

\textbf{Aloha Medicinals}

\begin{quote}
\textbf{url:} http://www.alohamedicinals.com/

\textbf{details:} innoculated grain, mushroom growing supplies
\end{quote}

\textbf{Arbico Organics}

\begin{quote}
\textbf{url:} http://www.arbico-organics.com/

\textbf{details:} EM•1®
\end{quote}

\textbf{Central Irrigation} (Elmsford)

\begin{quote}
\textbf{url:} http://www.centralirrigationsupply.com/

\textbf{details:} general irrigation supplies
\end{quote}

\textbf{Compostwerks}

\begin{quote}
\textbf{url:} http://www.compostwerks.com/

\textbf{details:} soil ammendments, biofertiliser ingredients

\textbf{contacts:} \emph{Gregg Twehues} (gregg@compostwerks.com);
\emph{Peter Schmidt} (peter@compostwerks.com)
\end{quote}

\textbf{Farmtek}

\begin{quote}
\textbf{url:} http://www.farmtek.com/

\textbf{details:} general equipment

\textbf{contact:} \emph{Virginia Daly} (vdaly@farmtek.com)
\end{quote}

\textbf{Growers Supply}

\begin{quote}
\textbf{url:} http://www.growerssupply.com/

\textbf{details:} general equipment
\end{quote}

\textbf{Johnny's Selected Seeds}

\begin{quote}
\textbf{url:} http://www.johnnyseeds.com/

\textbf{details:} seed, tools
\end{quote}

\textbf{Logan Labs}

\begin{quote}
\textbf{url:} http://www.loganlabs.com/

\textbf{details:} soil testing
\end{quote}

\textbf{Nolts Produce}

\begin{quote}
\textbf{url:} http://www.noltsproducesupplies.net/

\textbf{details:} row cover, irrigation supplies, tools
\end{quote}

\textbf{O2 Compost}

\begin{quote}
\textbf{url:} http://www.o2compost.com/

\textbf{details:} composting support

\textbf{contact:} \emph{Peter Moon} (peter@o2compost.com)
\end{quote}

\textbf{Teraganix}

\begin{quote}
\textbf{url:} http://www.teraganix.com/

\textbf{details:} EM•1®, pH test papers
\end{quote}

\textbf{The Green Growler} (Croton)

\begin{quote}
\textbf{url:} http://thegreengrowler.com/

\textbf{details:} airlocks
\end{quote}

\subsection{Glossary}\label{glossary}

\textbf{50\% moisture} - Compost or composting materials with
approximately 50\% moisture will make your hand moist when squeezed of
produce a drop or two of water.

\textbf{high carbon materials} - organic materials such as paper,
napkins, cardboard, woodships, straw etc. which have relatively much
more elemental carbon (C) than elemental nitrogen (N).

\textbf{high nitrogen materials} - organic materials such as manure,
vegetable scraps, grass clippings etc. which have high levels of
elemental nitrogen (N) relative to elemental carbon (C).
