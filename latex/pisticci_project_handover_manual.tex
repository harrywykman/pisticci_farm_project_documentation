\section{The Pisticci Farm Project}\label{the-pisticci-farm-project}

The Pisticci Farm Project is a project of the Upper Manhattan restaurant
\href{http://www.pisticcinyc.com/}{Pisticci}. The project has three
elements:

\begin{itemize}
\itemsep1pt\parskip0pt\parsep0pt
\item
  \emph{engaging} the \textbf{restaurant} staff and patrons
\item
  creating \emph{zero waste} through
  \hyperref[compost]{\textbf{composting}} within the city
\item
  \emph{producing} high-quality vegetables for Pisticci on the
  \hyperref[farm]{\textbf{multi-site farm}}
\end{itemize}

This document outlines the technologies and processes involved in the
composting and farming elements of the project.

\hyperdef{}{compost}{\subsection{Compost}\label{compost}}

The discarded organic material from Pisticci restaurant are separated at
source (ie. the kitchen, bar, etc.) from other `waste'. These materials
are composted using an oxygenated hot composting method to ensure rapid
decomposition, no unpleasant odours and the creation of a high quality
fertiliser and biological innoculum for the soil at the
\hyperref[farm]{farm}.

\subsubsection{Aerated Static Pile (ASP)
Composting}\label{aerated-static-pile-asp-composting}

Pisticci is using the aerated static pile (ASP) method of composting as
developed by Peter Moon of \href{http://www.o2compost.com}{O2 Compost}.

The Pisticci ASP system using includes three enclosed boxes which are
bottem-fed pressurised air through \emph{100 mm} (\emph{4 in}) from a
blower mounted above the bins.

\paragraph{Organic Material Collection
Process}\label{organic-material-collection-process}

\begin{itemize}
\itemsep1pt\parskip0pt\parsep0pt
\item
  Organic materials are collected in white plastic bags by the
  restaurant staff and deposited in wheelie-bins next to the compost
  bins.
\end{itemize}

\paragraph{Composting Processes}\label{composting-processes}

\textbf{Establishing a new Pile}

\begin{enumerate}
\def\labelenumi{\arabic{enumi}.}
\itemsep1pt\parskip0pt\parsep0pt
\item
  Ensure that all pipes in the manifold in the bottom of the bin are
  present and properly fitted together.
\item
  Add woodchips to just cover the pipes and thoroughly moisten the
  woodchips.
\item
  Open the valve above the bin to ensure that air will flow into the
  pile while composting proceeds.
\end{enumerate}

Organic materials may now be added.

\textbf{Ongoing Addition of Organic Materials}

\begin{enumerate}
\def\labelenumi{\arabic{enumi}.}
\itemsep1pt\parskip0pt\parsep0pt
\item
  Use the composting fork to mix through any previously added materials
  and add water from the hose to bring the moisture levels up to
  approximately \hyperref[50moisture]{50\% moisture}.
\item
  Empty two or three white bags of organic materials into the bin
  currently in use. Attempt to mix bags which are mostly vegetable waste
  (\hyperref[highnitrogen]{high nitrogen materials}) and bags which are
  mostly napkins / dry `\hyperref[highcarbon]{high carbon}' materials.
\item
  Mix these materials together in the bin using the composting fork.
\item
  Use the bucket to carry several buckets of woodchips from the woodchip
  storge bin to the compost bin to cover the material from the bags.
\item
  Mix the woodchips into the other materials using the composting fork.
\item
  Allow the materials to sit in the bin until the next addition to
  absorb moisture before adding any additional water (see point 1
  above).
\end{enumerate}

\textbf{When a Bin is Full}

\begin{enumerate}
\def\labelenumi{\arabic{enumi}.}
\itemsep1pt\parskip0pt\parsep0pt
\item
  When a bin is full, make sure the moisture levels are appropriate (ie.
  \hyperref[50moisture]{50\% moisture}), add a layer of woodchip over
  the top layer and cover with geotextile fabric / weed matting.
\item
  ensure the valve for the air is in the fully open position.
\item
  Leave to mature for as long as possible before the compost is taken up
  to the farm
\item
  Compost will be taken to the farm when two bins are full and one bin
  is one quarter to half-way full.
\end{enumerate}

\textbf{Transporting Compost to the Farm}

When two bins are full and one bin is one quarter to half-way full, the
most mature pile should be transported to the farm.

\begin{enumerate}
\def\labelenumi{\arabic{enumi}.}
\itemsep1pt\parskip0pt\parsep0pt
\item
  Position the truck somewhere near to the front of the restaurant which
  will allow for wheelbarrow access via the gate at the side of the
  building.
\item
  Lift the front panel of the bin to be removed so the the bolts which
  secure it are lifted from their holes and set the front panel aside.
\item
  Using a wheelbarrow, transport the compost to the truck until the bin
  is empty.
\item
  Clean all areas and close the bin by returning the front panel to its
  place.
\end{enumerate}

\paragraph{System Specifications}\label{system-specifications}

\begin{itemize}
\itemsep1pt\parskip0pt\parsep0pt
\item
  \textbf{The Bins} are 5' long, 4' high and 3' wide to fit within the
  small alleyway behind the restaurant.
\item
  \textbf{The Blower}
\item
  \textbf{The Timer} is set to be on for 2 minutes every 30 minutes to
  ensure the piles remain oxygenated.
\end{itemize}

\hyperdef{}{farm}{\subsection{Farm}\label{farm}}

The Pisticci Farm Project is a multi-site micro farm. At the time of
writing the farm consists of two sites:

\begin{itemize}
\itemsep1pt\parskip0pt\parsep0pt
\item
  43 Old Post Rd South, Croton-on-Hudson, NY (OPRS); and
\item
  716 Kitchawan Rd, Ossining, NY (KITCH).
\end{itemize}

The \textbf{Old Post Rd South} site was developed and farmed in 2015.
The \textbf{Kitchawan} site was secured at the end of 2015 and will be
developed and farmed in addition to the OPRS site in 2016.

\subsubsection{Old Post Rd South}\label{old-post-rd-south}

The Old Post Rd South site is the flagship site of the Pisticci Farm
Project. It consists of indoor and outdoor growing space. There are two
large old glass greenhouses which have been restored and developed for
all sesason production of vegetables. The 2015 growing season was used
for diverse crop production with a focus on fast growing, high value
crops, variety trials and winter greens production. In 2016, this site
will be used primarily for all season greens production and greenhouse
production of tomatoes, peppers, eggplant and cucumbers.

\paragraph{The Outdoor Growing Area}\label{the-outdoor-growing-area}

The outdoor growing area consists of 27 beds of varying sizes in three
blocks (Lower, Middle, Top). The beds are all 30 inches wide with 18
inch paths between them and vary in size from 10 to 40 feet with a total
of 620 linear feet of bed space (1550 ft2).

\paragraph{The Greenhouses}\label{the-greenhouses}

In 2015, the indoor space consisted of 12 beds across two greenhouses.
The beds ranged in size from 18 to 60 feet with a total of 356 linear
feet of bed space (890 ft2). In early 2016, these indoor beds were
delineated with 6" x 2" boards to create a total of 14 beds 29 inches
wide with 11 inch intermediate paths. This new configuration has a total
of XXX bed feet (XXX 1550 ft2).

\textbf{Hydronic Heating} is used in the greenhouses to allow for all
season production. The original soil is separated from the imported soil
and compost by 2 inches of foam insulation. PEX tubing runs over the top
of the insulation at 12 inch spacing. The PEX tubing is covered with
between 12 and 18 inches of topsoil mixed with compost. This soil is
kept at a minimum of 50°F (10°C) to allow for ongoing plant growth
during the cooler month.

\subsubsection{Kitchawan}\label{kitchawan}

Kitchawan Farm is a diversified working farm with a focus on rough
housing of horses. The Pisticci Farm project leases a small area for
vegetable production. In Spring of 2016, this area will be developed
into sets of 30 inch wide beds. This site will be used for crops n

\subsubsection{Fertility}\label{fertility}

Soil is at the heart of any growing system with integrity. The Pisticci
Farm project utilises a number of strategies for developing and
maintaining a healthy soil for the production of healthy nutritios
plants. These include:

\begin{itemize}
\itemsep1pt\parskip0pt\parsep0pt
\item
  composting
\item
  soil testing and ammending for mineral balance
\item
  biological innocula / biofertiliser
\item
  minimal tillage
\item
  fertigation
\end{itemize}

\paragraph{Soil Testing and Ammending for Mineral
Balance}\label{soil-testing-and-ammending-for-mineral-balance}

\paragraph{Biological Innocula}\label{biological-innocula}

\paragraph{Fertigation}\label{fertigation}

\subsection{Suppliers}\label{suppliers}

\begin{longtable}[c]{@{}lll@{}}
\toprule\addlinespace
Supplier & Description & Contact(s)
\\\addlinespace
\midrule\endhead
\textbf{Aloha Medicinals} http://www.alohamedicinals.com/ & innoculated
grain, mushroom growing supplies
\\\addlinespace
\textbf{Arbico Organics} http://www.arbico-organics.com/ & EM-1 &
\\\addlinespace
\textbf{Central Irrigation} (Elmsford)
http://www.centralirrigationsupply.com/ & general irrigation supplies
\\\addlinespace
\textbf{Compostwerks} http://www.compostwerks.com/ & soil ammendments,
biofertiliser ingredients & \emph{Gregg Twehues}
(gregg@compostwerks.com) \emph{Peter Schmidt} (peter@compostwerks.com)
\\\addlinespace
\textbf{Farmtek} http://www.farmtek.com/ & general equipment &
\emph{Virginia Daly} (vdaly@farmtek.com)
\\\addlinespace
\textbf{Growers Supply} http://www.growerssupply.com/ & general
equipment
\\\addlinespace
\textbf{Johnny's Selected Seeds} http://www.johnnyseeds.com/ & seed,
tools
\\\addlinespace
\textbf{Logan Labs} & soil testing
\\\addlinespace
\textbf{Nolts Produce} http://www.noltsproducesupplies.net/ & row cover,
irrigation supplies, tools
\\\addlinespace
\textbf{O2 Compost} http://www.o2compost.com/ & composting support &
\emph{Peter Moon} (peter@o2compost.com)
\\\addlinespace
\bottomrule
\end{longtable}

\subsection{Glossary}\label{glossary}

\textbf{50\% moisture} - Compost or composting materials with
approximately 50\% moisture will make your hand moist when squeezed of
produce a drop or two of water.

\textbf{high carbon materials} - organic materials such as paper,
napkins, cardboard, woodships, straw etc. which have relatively much
more elemental carbon (C) than elemental nitrogen (N).

\textbf{high nitrogen materials} - organic materials such as manure,
vegetable scraps, grass clippings etc. which have high levels of
elemental nitrogen (N) relative to elemental carbon (C).
