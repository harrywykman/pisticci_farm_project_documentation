\section{The Pisticci Farm Project}\label{the-pisticci-farm-project}

The Pisticci Farm Project is a project of the Upper Manhattan restaurant
\href{http://www.pisticcinyc.com/}{Pisticci}. The project has three
elements:

\begin{itemize}
\tightlist
\item
  \emph{engaging} the \textbf{restaurant} staff and patrons
\item
  creating \emph{zero waste} through
  \protect\hyperlink{compost}{\textbf{composting}} within the city
\item
  \emph{producing} high-quality vegetables for Pisticci on the
  \protect\hyperlink{farm}{\textbf{multi-site farm}}
\end{itemize}

This document outlines the technologies and processes involved in the
composting and farming elements of the project.

\hypertarget{compost}{\subsection{Compost}\label{compost}}

The discarded organic material from Pisticci restaurant are separated at
source (ie. the kitchen, bar, etc.) from other `waste'. These materials
are composted using an oxygenated hot composting method to ensure rapid
decomposition, no unpleasant odours and the creation of a high quality
fertiliser and biological innoculum for the soil at the
\protect\hyperlink{farm}{farm}.

\subsubsection{Aerated Static Pile (ASP)
Composting}\label{aerated-static-pile-asp-composting}

Pisticci is using the aerated static pile (ASP) method of composting as
developed by Peter Moon of \href{http://www.o2compost.com}{O2 Compost}.

The Pisticci ASP system using includes three enclosed boxes which are
bottem-fed pressurised air through \emph{100 mm} (\emph{4 in}) from a
blower mounted above the bins.

\paragraph{Organic Material Collection
Process}\label{organic-material-collection-process}

\begin{itemize}
\tightlist
\item
  Organic materials are collected in white plastic bags by the
  restaurant staff and deposited in wheelie-bins next to the compost
  bins.
\end{itemize}

\paragraph{Composting Processes}\label{composting-processes}

\textbf{Establishing a new Pile}

\begin{enumerate}
\def\labelenumi{\arabic{enumi}.}
\tightlist
\item
  Ensure that all pipes in the manifold in the bottom of the bin are
  present and properly fitted together.
\item
  Add woodchips to just cover the pipes and thoroughly moisten the
  woodchips.
\item
  Open the valve above the bin to ensure that air will flow into the
  pile while composting proceeds.
\end{enumerate}

Organic materials may now be added.

\textbf{Ongoing Addition of Organic Materials}

\begin{enumerate}
\def\labelenumi{\arabic{enumi}.}
\tightlist
\item
  Empty two or three white bags of organic materials (attempt to mix
  bags which are mostly vegetable waste
  (\protect\hyperlink{highnitrogen}{high nitrogen materials}) and bags
  which are mostly napkins / dry `\protect\hyperlink{highcarbon}{high
  carbon}' materials
\end{enumerate}

\paragraph{System Specifications}\label{system-specifications}

\begin{itemize}
\tightlist
\item
  \textbf{The Bins} are 5' long, 4' high and 3' wide to fit within the
  small alleyway behind the restaurant.
\item
  \textbf{The Blower}
\item
  \textbf{The Timer} is set to be on for 2 minutes every 30 minutes to
  ensure the piles remain oxygenated.
\end{itemize}

\hypertarget{farm}{\subsection{Farm}\label{farm}}

\subsection{Glossary}\label{glossary}

\textbf{high carbon materials} - organic materials such as paper,
napkins, cardboard, woodships, straw etc. which have relatively much
more elemental carbon (C) than elemental nitrogen (N).

\textbf{high nitrogen materials} - organic materials such as manure,
vegetable scraps, grass clippings etc. which have high levels of
elemental nitrogen (N) relative to elemental carbon (C).
